\section{Related work}
\label{sec:related_work}

This project is inspired by PrEsto~\cite{PrEsto}, which accelerates power measurement using FPGA. 
It generates linear power models using linear regression, and then the model is mapped onto FPGA. 
However, designers should provide the important signals to the power model generators, and the power calculation logics are hardcoded in FPGA. 
Our approach differs in that the power model generator selects the important signals and we employ a counter-based method to calculate power consumption of the systems.

Bogliolo \textit{et al.}~\cite{Bogliolo2000} suggests regression-based RTL power modeling. 
They exploit linear regression to approximate the design's power consumption with RTL signal activities.
However, this approach cannot be applied to complex designs because it does not reduce the signal list to be watched.
Moreover, it only considers the signals in the border of macro blocks, which attributes to its inaccuracy.

There are several papers about FPGA-accelerated simulations.~\cite{Protoflex,Fast,RAMPGold,FAME,HAsim}
These works show that using FPGAs can reduce the architectural simulation dramatically.
Our project also uses FPGA simulation to accelerated power analysis

Isci \textit{et al.}~\cite{Isci2003} suggests a counter-based method to calculate the system’s power. 
However, it is limited to the real processor, Pentium 4, and thus, not flexible for any other processors. 
Moreover, it has to read the fixed activity counters to calculate power numbers. 
Our approach can be applied to general processor designs and generates activity counters for any important signals.

There are a bunch of attempts to accelerate power measurement of DSP designs~\cite{Coburn2005,Atienza2006,Ghodrat2007}.
However, these are not scalable, so it cannot be applied to complex designs' power measurement.

Bhattacharjee \textit{et al.}~\cite{Bhattacharjee2008} proposed a counter-based method using FPGA emulation. 
This method counts specific events of the processor to calculate power, which requires architectural intuition. 
For this reason, it is not applicable to general logic designs in contrast to our approach.